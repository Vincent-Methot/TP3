\documentclass[a4paper]{article}
\usepackage[x11names,svgnames]{xcolor}
\usepackage[T1]{fontenc}
\usepackage[utf8]{inputenc}
\usepackage[francais]{babel}
\usepackage{amsmath}
\usepackage{graphicx}
\usepackage{subfigure}
\usepackage[colorinlistoftodos]{todonotes}
\usepackage{array}
\usepackage{setspace}
\usepackage{fullpage}
\usepackage[justification=centering]{caption} % necessaire pour caption longue de plus d'une ligne.
\usepackage{hyperref} 
%\pdfcompresslevel=9 
\hypersetup{ 
     colorlinks=true, %colorise les liens 
     breaklinks=true, %permet le retour àˆ la ligne dans les liens trop longs 
     urlcolor= blue,  %couleur des hyperliens 
     linkcolor= Blue4, %couleur des liens internes crééŽs, 
}
\usepackage{listings}
\lstset{
language=python,
commentstyle=\textit,
basicstyle=\ttfamily\color{Black},
keywordstyle=\color{DarkRed},
commentstyle=\color{Blue3}\normalfont,
}
\newcommand{\cin}[1]{\lstinline{#1}}

% Francisation
\addto\captionsfrench{\renewcommand{\tablename}{{\scshape Tab.}}}

% Colonnes de largeur fixe
\newcolumntype{L}[1]{>{\raggedright\let\newline\\\arraybackslash\hspace{0pt}}m{#1}}
\newcolumntype{C}[1]{>{\centering\let\newline\\\arraybackslash\hspace{0pt}}m{#1}}
\newcolumntype{R}[1]{>{\raggedleft\let\newline\\\arraybackslash\hspace{0pt}}m{#1}}


\def\thesection{Question \arabic{section}}
\def\thesubsection{\alph{subsection}}

\onehalfspacing

% Bibliographie
\bibliographystyle{ieeetr}


\title{TP3 - IMN530}

\author{FOUQUET, Jérémie et MÉTHOT, Vincent}

\date{28 avril 2014}

\begin{document}
\maketitle

\section{IRM fonctionnelle}

Plusieurs outils d'analyse existent déjàpour traiter des données d'IRMf. Nous avons choisis de les utiliser directement plutôt que de les implémenter en python. Deux suites logicielles ont retenu notre intérêt (puisque nous les connaissons déjà), soit FSL [http://fsl.fmrib.ox.ac.uk/fsl/fslwiki/] et AFNI [http://afni.nimh.nih.gov/], qu'il faudra avoir installé pour faire fonctionner le script associé à ce numéro (Q1\_IRMf.sh).

\subsection{Étapes de reconstruction}

Il faut garder à l'esprit qu'à chaque étape de reconstruction, il est fortement conseillé d'inspecter visuellement les données. Dès leur réception, on a visuellement inspecté plusieurs tranches de \emph{fmri.nii} à tous les temps pour s'assurer que les artéfacts n'étaient pas trop important et que la correction de mouvement n'était pas nécessaire (voir Fig. \ref{fmri_anatomist}), comme mentionné dans la question. De plus, nous avons effectué une transformée de Fourier des séries temporelles (voir Fig. \ref{fmri_fft}).

\begin{figure}[h]
   \caption{\label{fmri_anatomist} Inspection visuelle de fmri.nii dans anatomist. On peut inspecter plusieurs tranches pour tous les points temporels à l'aide des deux curseurs à droite, comme dans un film.}
   \centering
   \includegraphics[width=\textwidth]{fmri_anatomist}
\end{figure}

\begin{figure}[h]
   \caption{\label{fmri_fft} Coupe sagitale de la transformée de Fourier de fmri.nii à chaque point. Une fréquence près de 0.02 Hz (la fréquence fondamentale du stimulus) est affichée dans AFNI. Les zones blanches peuvent être dues à de l'activation cérébrale mais aussi (et surtout) à un effet de bord relié à du mouvement.}
   \centering
   \includegraphics[width=\textwidth]{fmri_fft}
\end{figure}

Les étapes de reconstruction suivantes furent appliquées sur les données d'IRMf. Le résultat de chacune est illustrée sur la Fig. \ref{fmri_reconstruction}.

\begin{itemize}
	\item[A] Lissage spatial : pour augmenter le rapport signal sur bruit, on procède à un débuitage gaussien de toute l'image.
	% Il y aurait avantage à utiliser un filtrage bilatéral, ou quelque chose de plus poussé...
	\item[B] Filtrage temporel : du bruit et des artéfacts de nature physiologique se retrouvent aux hautes fréquences. Nous avous décidé de couper les fréquences supérieures à 0.06 Hz (sachant que notre signal idéal à un fréquence fondamentale de 0.02 Hz) de chacune de nos séries temporelles. 
	\item[C] Corrélation point à point : on corrèle le signal dans chaque voxel avec le signal idéal (contenu dans le fichier Data/ideal.txt) pour en tirer une carte de coefficients de corrélatoin entre -1 et 1. On pourra ensuite seuiller cette carte pour obtenir un masque binaire des zones d'activations.
\end{itemize}

\begin{figure}[h]
   \caption{\label{fmri_reconstruction} A - Image originelle. B - Séries temporelles pour quelques voxels adjacents de l'image originelle. C - Image débruitée (convolution avec une gaussienne de FWHM = 6mm). D - Série temporelle débruitée. E - Image après passage d'une filtre passe-bas. La valeur moyenne (fréquence nulle) représentant l'information structurelle est perdue. F - Série temporelle de l'image filtrée. G - Corrélation point à point (seuillé à $\rho > 0.5$). H - Séries temporelles pour quelques voxels adjacents de l'image reconstruite. Position du point sélectionné: x = 22, y = 13, z = 24. Les séries temporelles rouge représentent le signal idée. Celles en noir représentent le signal mesuré/transformées.}
   \centering
   \includegraphics[height=0.85\textheight]{fmri_reconstruction}
\end{figure}


\subsection{Segmentation}

\subsection{Zones d'activation}

\section{IRM de diffusion}

\subsection{Estimation des tenseurs}

\subsection{FA et ADC}

\subsection{Tractographie}

\section{Fusion}

\subsection{Justification}

\subsection{Connectivité des zones fonctionnelles}

\section{Bonus}

\subsection{FA et ADC}

\subsection{Tractographie avec Dipy}


\end{document}